\documentclass[hyperref={pdfstartview=Fit}]{beamer}
%\documentclass[hyperref={pdfstartview=Fit,pdfpagemode=FullScreen}]{beamer}
\mode<presentation>%
{
  \usetheme{Warsaw}
  \setbeamercovered{transparent}
}
\usepackage[english]{babel}
\usepackage[utf8]{inputenc}

\usepackage{lmodern}
\usepackage[T1]{fontenc}

\usepackage[lined, boxed]{algorithm2e}

\newcommand{\imsize}{}
\providecommand{\abs}[1]{\left\lvert#1\right\rvert}

\newlength{\widthtmp}
\newcommand{\getWidth}[1]{%
  \settowidth{\widthtmp}{#1}%
  \the\widthtmp%
}

% Delete this, if you do not want the table of contents to pop up at
% the beginning of each subsection:
% \AtBeginSubsection[]
% {
%   \begin{frame}<beamer>{Outline}
%     \tableofcontents[currentsection,currentsubsection]
%   \end{frame}
% }


% If you wish to uncover everything in a step-wise fashion, uncomment
% the following command:

%\beamerdefaultoverlayspecification{<+->}

\title[Advection equation and antidiffusion]%
{Antidiffusion techniques to refine the numerical solution of the advection equation}

\subtitle{Case study: Smolarkiewicz' iterative approach}

\author[Burger, Wolterink]%
{Gerhard Burger \and Jelmer Wolterink}

\institute[Utrecht University]%
{
  Scientific Computing\\
  Department of Mathematics\\
  Utrecht University
}

\date[31-Oct-2011] %
{Project presentations SOAC, 31 October 2011}

\subject{Advection equation and antidiffusion}

 \pgfdeclareimage[height=0.5cm]{university-logo}{UU_logo_fullcolor_uncoated_sol_left}
 \logo{\pgfuseimage{university-logo}}

\begin{document}

\begin{frame}
  \titlepage
\end{frame}

\begin{frame}{Outline}
  \tableofcontents
  % You might wish to add the option [pausesections]
\end{frame}

\section{The advection equation}
\subsection{Importance}

\begin{frame}{Usage examples}
\begin{itemize}
   \item everywhere...
\end{itemize}

\end{frame}


\subsection{Diffusion}

\begin{frame}
\frametitle{Example of diffusion with an upstream scheme}
\begin{figure}
\includegraphics<1>{animation/anim-0}%
\includegraphics<2>{animation/anim-1}%
\includegraphics<3>{animation/anim-2}%
\includegraphics<4>{animation/anim-3}%
\includegraphics<5>{animation/anim-4}%
\includegraphics<6>{animation/anim-5}%
\includegraphics<7>{animation/anim-6}%
\includegraphics<8>{animation/anim-7}%
\includegraphics<9>{animation/anim-8}%
\includegraphics<10>{animation/anim-9}%
\end{figure}
\end{frame}

\subsection{Antidiffusion methods}

\begin{frame}
\frametitle{Antidiffusion methods 1}
\begin{itemize}
\item Flux-corrected transport (FCT) method (Boris and Book, 1973),
\item self-adjusting hybrid scheme (SAHS) (Harten en Zwas, 1972),
\end{itemize}
   both can be very accurate but require excessive computing time, better is the
\begin{itemize}
   \item hybrid-type scheme based on a Crowley advection scheme (Clark and Hall, 1979),
\end{itemize}
   with more diffusion but half the computation time.
\end{frame}

\begin{frame}
\frametitle{Antidiffusion methods 2}
\begin{itemize}
\item Smolarkiewicz' iterative correction
\end{itemize}
   less time consuming while results are comparable to those of the more complex hybrid schemes
\end{frame}

\section{Case study: Smolarkiewicz}
\subsection{Analyzing the scheme}

\begin{frame}
\frametitle{Basis: upstream on a staggered grid}
We start with the following upstream advection equation on staggered grid:
\begin{equation*}
 \psi_i^{N+1} = \psi_i^N - \Big( F \left( \psi_i^N,\psi_{i+1}^N,u_{i+1/2}^N\right)
-F \left( \psi_{i-1}^N,\psi_{i}^N,u_{i-1/2}^N\right) \Big),
\end{equation*}
where
\begin{multline*}
F \left( \psi_i^N,\psi_{i+1}^N,u_{i+1/2}^N\right) =\\
\Big( \left( u_{i+1/2}^N + \abs{u_{i+1/2}^N} \right) \psi_i^N
+ \left( u_{i+1/2}^N - \abs{u_{i+1/2}^N} \right) \psi_{i+1}^N \Big)
\frac{\Delta t}{2 \Delta x}.
\end{multline*}
\end{frame}

\begin{frame}
\frametitle{Method of lines 1}
Writing the scheme out and collecting terms gives
\begin{equation*}
\begin{split}
\psi_i^{N+1} &=
\frac{\Delta t}{2 \Delta x} \left( u_{i-1/2}^N + \abs{u_{i-1/2}^N} \right) \psi_{i-1}^N\\
&+ \left(1 - \frac{\Delta t}{2 \Delta x} \left( u_{i+1/2}^N + \abs{u_{i+1/2}^N} - u_{i-1/2}^N + \abs{u_{i-1/2}^N} \right) \right) \psi_i^N\\
&-\frac{\Delta t}{2 \Delta x} \left( u_{i+1/2}^N - \abs{u_{i+1/2}^N} \right) \psi_{i+1}^N
\end{split}
\end{equation*}
\end{frame}

\begin{frame}
\frametitle{Method of lines 2}
This can be rewritten to 
\begin{equation*}
\psi_i^{N+1} = \alpha_i \psi_{i-1}^N + \beta_i \psi_i^N +\gamma_i \psi_{i+1}^N, \quad \text{for } i=1,\ldots,M-1,
\end{equation*}
 where we have that
\begin{align*}
\alpha_i &= \frac{\Delta t}{2 \Delta x} \left( u_{i-1/2}^N + \abs{u_{i-1/2}^N} \right),\\
 \beta_i &= \left(1 - \frac{\Delta t}{2 \Delta x} \left( u_{i+1/2}^N + \abs{u_{i+1/2}^N} - u_{i-1/2}^N + \abs{u_{i-1/2}^N} \right) \right),\\
\gamma_i &= -\frac{\Delta t}{2 \Delta x} \left( u_{i+1/2}^N - \abs{u_{i+1/2}^N} \right).
\end{align*}
\end{frame}

\begin{frame}
\frametitle{Method of lines 3}
We can also write this in matrix form using Direchlet boundary conditions
\begin{itemize}
\item \only<1>\alert{$y(0)=0$ and $y(M)=0$}
\item \only<2>\alert{using periodic boundary conditions $y(0)=y(M)$}
\end{itemize}
\begin{equation*}
\begin{bmatrix}\psi_{1}^{N+1}\\\psi_{2}^{N+1}\\ \vdots \\\psi_{M-2}^{N+1}\\\psi_{M-1}^{N+1}\end{bmatrix} =
\begin{bmatrix}\beta_1&\gamma_1&&&\alert{\only<1>{0}\only<2>{\alpha_1}}\\ \alpha_2&\beta_2&\gamma_2\\ &\ddots&\ddots&\ddots\\&&\alpha_{M-2}&\beta_{M-2}&\gamma_{M-2}\\\alert{\only<1>{0}\only<2>{\gamma_{M-1}}}&&&\alpha_{M-1}&\beta_{M-1}\\ \end{bmatrix}
\begin{bmatrix}\psi_{1}^N\\\psi_{2}^N\\ \vdots \\\psi_{M-2}^N\\\psi_{M-1}^N\end{bmatrix}
\end{equation*}
So the new values of $\psi$ can be obtained by a sparse matrix-vector multiplication 
\end{frame}

\begin{frame}
\frametitle{Antidiffusion 1}
To apply antidiffusion we need to redefine the scheme into
\begin{align*}
 \psi_i^{*} &= \psi_i^N - \Big( F \left( \psi_i^N,\psi_{i+1}^N,u_{i+1/2}^N\right)
-F \left( \psi_{i-1}^N,\psi_{i}^N,u_{i-1/2}^N\right) \Big),\\
 \psi_i^{N+1} &= \psi_i^* - \Big( F \left( \psi_i^*,\psi_{i+1}^*,\tilde{u}_{i+1/2}^N\right)
-F \left( \psi_{*}^N,\psi_{*}^N,\tilde{u}_{i-1/2}^N\right) \Big),\
\end{align*}
were the antidiffusion velocity $\tilde{u}_{i+1/2}$ is defined as
\begin{equation*}
\tilde{u}_{i+1/2} = \frac{\left(\abs{u_{i+1/2}}\Delta x - \Delta t u_{i+1/2}^2 \right) \left( \psi_{i+1}^*-\psi_i^*\right)}{ \left( \psi_i^*+\psi_{i+1}^*+\epsilon \right) \Delta x}
\end{equation*}
\end{frame}

\begin{frame}
\frametitle{Antidiffusion 2}
The Method of lines can be applied in the same way to the second step in the scheme,\\
 so the whole scheme can be executed with two sparse matrix-vector multiplications and a matrix update.
\end{frame}

\subsection{Implementation}
\begin{frame}
\frametitle{Algorithm}
\begin{algorithm}[H]
\SetKwData{Iter}{iter}\SetKwData{Mone}{mat1}\SetKwData{Mtwo}{mat2}
\SetKwData{Steps}{tsteps}
\SetKwFunction{MMat}{ComputeMatrix}\SetKwFunction{MMult}{MatrixMultiplication}
\SetKwFunction{CADV}{ComputeAntidiffusionVelocity}
\SetKwInOut{Input}{input}\SetKwInOut{Output}{output}
\Input{Initial configuration $\psi^0$, velocities $u$, number of iterations \Iter} 
\Output{Final configuration $\psi^N$}
\BlankLine
$\psi \leftarrow \psi^0$\;
\Mone $\leftarrow$ \MMat{$u$}\;
\For{$i \leftarrow 0$ \KwTo \Iter}{
$\psi \leftarrow$ \MMult{\Mone,$\psi$}\;
\For{$j \leftarrow 1$ \KwTo \Iter}{
$\widetilde{u} \leftarrow$ \CADV{$\psi,u$}\;
\Mtwo $\leftarrow$ \MMat{$\widetilde{u}$}\;
$\psi \leftarrow$ \MMult{\Mtwo,$\psi$}\;
}
}
\end{algorithm}
\end{frame}

\subsection{numerical results}

%
% % All of the following is optional and typically not needed.
% \appendix
% \section<presentation>*{\appendixname}
% \subsection<presentation>*{For Further Reading}
%
% \begin{frame}[allowframebreaks]
%   \frametitle<presentation>{For Further Reading}
%
%   \begin{thebibliography}{10}
%
%   \beamertemplatebookbibitems
%   % Start with overview books.
%
%   \bibitem{Author1990}
%     A.~Author.
%     \newblock {\em Handbook of Everything}.
%     \newblock Some Press, 1990.
%
%
%   \beamertemplatearticlebibitems
%   % Followed by interesting articles. Keep the list short.
%
%   \bibitem{Someone2000}
%     S.~Someone.
%     \newblock On this and that.
%     \newblock {\em Journal of This and That}, 2(1):50--100,
%     2000.
%   \end{thebibliography}
% \end{frame}

\end{document}


