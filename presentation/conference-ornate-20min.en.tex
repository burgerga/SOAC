\documentclass[hyperref={pdfstartview=Fit}]{beamer}
%\documentclass[hyperref={pdfstartview=Fit,pdfpagemode=FullScreen}]{beamer}
\mode<presentation>
{
  \usetheme{Warsaw}
  \setbeamercovered{transparent}
}
\usepackage[english]{babel}
\usepackage[utf8]{inputenc}

\usepackage{lmodern}
\usepackage[T1]{fontenc}

\title[Advection equation and antidiffusion]
{Using antidiffusion to solve the advection equation more accurately}

\subtitle{Case study: Smolarkiewicz' iterative approach}

\author[Burger, Wolterink]
{Gerhard Burger \and Jelmer Wolterink}

\institute[Utrecht University]
{
  Scientific Computing\\
  Department of Mathematics\\
  Utrecht University
}

\date[31-Oct-2011] %
{Project presentations SOAC, 31 October 2011}

\subject{Advection equation and antidiffusion}

 \pgfdeclareimage[height=0.5cm]{university-logo}{UU_logo_fullcolor_uncoated_sol_left}
 \logo{\pgfuseimage{university-logo}}



% Delete this, if you do not want the table of contents to pop up at
% the beginning of each subsection:
% \AtBeginSubsection[]
% {
%   \begin{frame}<beamer>{Outline}
%     \tableofcontents[currentsection,currentsubsection]
%   \end{frame}
% }


% If you wish to uncover everything in a step-wise fashion, uncomment
% the following command:

%\beamerdefaultoverlayspecification{<+->}
\newcommand{\imsize}{}

\begin{document}

\begin{frame}
  \titlepage
\end{frame}

\begin{frame}{Outline}
  \tableofcontents
  % You might wish to add the option [pausesections]
\end{frame}

\section{The advection equation in modeling}
\subsection{Examples of use}
\subsection{Diffusion}

\section{Antidiffusion methods}

\section{Case study: Smolarkiewicz}
\subsection{the scheme and method of lines}
\subsection{implementation}
\subsection{numerical results}

\begin{frame}
\frametitle{The scheme}
We start with the following upstream advection equation on staggered grid:
\begin{equation*}
 \psi_i^{N+1} = \psi_i^N - \Big( F \left( \psi_i^N,\psi_{i+1}^N,u_{i+1/2}^N\right)
-F \left( \psi_{i-1}^N,\psi_{i}^N,u_{i-1/2}^N\right) \Big),
\end{equation*}
where
% \begin{multline*}
% F \left( \psi_i^N,\psi_{i+1}^N,u_{i+1/2}^N\right) =\\
% \Big( \left( u_{i+1/2}^N + \abs{u_{i+1/2}^N} \right) \psi_i^N
% + \left( u_{i+1/2}^N - \abs{u_{i+1/2}^N} \right) \psi_{i+1}^N \Big)
% \frac{\Delta t}{2 \Delta x}.
% \end{multline*}
\end{frame}

\begin{frame}
\renewcommand{\imsize}{0.5\textwidth}
\begin{figure}
        \includegraphics<1>[width=\imsize]{animation/anim0.pdf}
        \includegraphics<2>[width=\imsize]{animation/anim1.pdf}
        \includegraphics<3>[width=\imsize]{animation/anim2.pdf}
        \includegraphics<4>[width=\imsize]{animation/anim3.pdf}
        \includegraphics<5>[width=\imsize]{animation/anim4.pdf}
        \includegraphics<6>[width=\imsize]{animation/anim5.pdf}
	\includegraphics<7>[width=\imsize]{animation/anim6.pdf}
	\includegraphics<8>[width=\imsize]{animation/anim7.pdf}
        \includegraphics<9>[width=\imsize]{animation/anim8.pdf}
	\includegraphics<10>[width=\imsize]{animation/anim9.pdf}
    \end{figure}
\end{frame}

% \begin{frame}{Writing it out}
% Inserting this and collecting terms gives us
% \begin{equation*}
% \begin{split}
% \psi_i^{N+1} &=
% \frac{\Delta t}{2 \Delta x} \left( u_{i-1/2}^N + \abs{u_{i-1/2}^N} \right) \psi_{i-1}^N\\
% &+ \left(1 - \frac{\Delta t}{2 \Delta x} \left( u_{i+1/2}^N + \abs{u_{i+1/2}^N} - u_{i-1/2}^N + \abs{u_{i-1/2}^N} \right) \right) \psi_i^N\\
% &-\frac{\Delta t}{2 \Delta x} \left( u_{i+1/2}^N - \abs{u_{i+1/2}^N} \right) \psi_{i+1}^N\\
% \end{split}
% \end{equation*}
% \end{frame}

% \begin{frame}{Writing it out}
% We can write this as
% \begin{equation*}
% \psi_i^{N+1} = \alpha_i \psi_{i-1}^N + \beta_i \psi_i^N +\gamma_i \psi_{i+1}^N, \quad \text{for } i=1,\ldots,M-1,
% \end{equation*}
%  where we have that
% \begin{align*}
% \alpha_i &= \frac{\Delta t}{2 \Delta x} \left( u_{i-1/2}^N + \abs{u_{i-1/2}^N} \right),\\
%  \beta_i &= \left(1 - \frac{\Delta t}{2 \Delta x} \left( u_{i+1/2}^N + \abs{u_{i+1/2}^N} - u_{i-1/2}^N + \abs{u_{i-1/2}^N} \right) \right),\\
% \gamma_i &= -\frac{\Delta t}{2 \Delta x} \left( u_{i+1/2}^N - \abs{u_{i+1/2}^N} \right).
% \end{align*}
% \end{frame}
%
%
% \begin{frame}{Matrix form}
% We can also write this in matrix form
% \begin{equation*}
% \begin{bmatrix}\psi_{1}^{N+1}\\\psi_{2}^{N+1}\\ \vdots \\\psi_{M-2}^{N+1}\\\psi_{M-1}^{N+1}\end{bmatrix} =
% \begin{bmatrix}\beta_1&\gamma_1\\ \alpha_2&\beta_2&\gamma_2\\ &\ddots&\ddots&\ddots\\&&\alpha_{M-2}&\beta_{M-2}&\gamma_{M-2}\\&&&\alpha_{M-1}&\beta_{M-1}\\ \end{bmatrix}
% \begin{bmatrix}\psi_{1}^N\\\psi_{2}^N\\ \vdots \\\psi_{M-2}^N\\\psi_{M-1}^N\end{bmatrix}
% \end{equation*}
%
%
% \end{frame}
%
% \begin{frame}{Make Titles Informative.}
%
%   You can create overlays\dots
%   \begin{itemize}
%   \item using the \texttt{pause} command:
%     \begin{itemize}
%     \item
%       First item.
%       \pause
%     \item
%       Second item.
%     \end{itemize}
%   \item
%     using overlay specifications:
%     \begin{itemize}
%     \item<3->
%       First item.
%     \item<4->
%       Second item.
%     \end{itemize}
%   \item
%     using the general \texttt{uncover} command:
%     \begin{itemize}
%       \uncover<5->{\item
%         First item.}
%       \uncover<6->{\item
%         Second item.}
%     \end{itemize}
%   \end{itemize}
% \end{frame}
%
%
% \subsection{Previous Work}
%
% \begin{frame}{Make Titles Informative.}
% \end{frame}
%
% \begin{frame}{Make Titles Informative.}
% \end{frame}
%
%
%
% \section{Our Results/Contribution}
%
% \subsection{Main Results}
%
% \begin{frame}{Make Titles Informative.}
% \end{frame}
%
% \begin{frame}{Make Titles Informative.}
% \end{frame}
%
% \begin{frame}{Make Titles Informative.}
% \end{frame}
%
%
% \subsection{Basic Ideas for Proofs/Implementation}
%
% \begin{frame}{Make Titles Informative.}
% \end{frame}
%
% \begin{frame}{Make Titles Informative.}
% \end{frame}
%
% \begin{frame}{Make Titles Informative.}
% \end{frame}
%
%
%
% \section*{Summary}
%
% \begin{frame}{Summary}
%
%   % Keep the summary *very short*.
%   \begin{itemize}
%   \item
%     The \alert{first main message} of your talk in one or two lines.
%   \item
%     The \alert{second main message} of your talk in one or two lines.
%   \item
%     Perhaps a \alert{third message}, but not more than that.
%   \end{itemize}
%
%   % The following outlook is optional.
%   \vskip0pt plus.5fill
%   \begin{itemize}
%   \item
%     Outlook
%     \begin{itemize}
%     \item
%       Something you haven't solved.
%     \item
%       Something else you haven't solved.
%     \end{itemize}
%   \end{itemize}
% \end{frame}
%
%
%
% % All of the following is optional and typically not needed.
% \appendix
% \section<presentation>*{\appendixname}
% \subsection<presentation>*{For Further Reading}
%
% \begin{frame}[allowframebreaks]
%   \frametitle<presentation>{For Further Reading}
%
%   \begin{thebibliography}{10}
%
%   \beamertemplatebookbibitems
%   % Start with overview books.
%
%   \bibitem{Author1990}
%     A.~Author.
%     \newblock {\em Handbook of Everything}.
%     \newblock Some Press, 1990.
%
%
%   \beamertemplatearticlebibitems
%   % Followed by interesting articles. Keep the list short.
%
%   \bibitem{Someone2000}
%     S.~Someone.
%     \newblock On this and that.
%     \newblock {\em Journal of This and That}, 2(1):50--100,
%     2000.
%   \end{thebibliography}
% \end{frame}

\end{document}


